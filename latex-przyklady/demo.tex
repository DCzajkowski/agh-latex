\documentclass[a4paper,12pt]{article}      % preambu�a
\usepackage[polish]{babel}
\usepackage[latin2]{inputenc} %utf8, cp1250
\usepackage[T1]{fontenc}
\usepackage{times}

\begin{document}                           % cz�� g��wna

\section{Sztuczne �ycie}


W~1987 roku na konferencji naukowej w Nowym Meksyku w Stanach Zjednoczonych narodzi�a si� kolejna dziedzina sztucznej inteligencji nazwana sztucznym �yciem (artificial life). Zajmuje si� ona tworzeniem system�w, kt�re mog� si� samodzielnie rozwija� i doskonali�. 

Ju� w 1968 roku Aristin Lindenmaier podj�� pr�b� stworzenia uniwersalnego j�zyka genetycznego u�ywanego przez ro�liny i~w~rezultacie rozwin�� algorytmy odtwarzaj�ce struktur� ro�lin. W~wyniku tych prac powsta� matematyczny opis wzrostu ro�lin, na cze�� naukowca nazwany L-systemem. 



\end{document}
